\documentclass[11pt]{article}
\usepackage[italian]{babel}
\usepackage[T1]{fontenc}
\usepackage[utf8x]{inputenc}
\usepackage{hyperref}
\usepackage{tikz}
\usepackage{tikz-uml}

\title{\textbf{SimpleSocial}\\Progetto di Fine Corso -- RCL}
\author{
	Francesco Landolfi\\
	\href{mailto:fran.landolfi@gmail.com}
	{\tt<fran.landolfi@gmail.com>}\\
	Matricola: 444151
}
\date{\today}
\begin{document}
\maketitle


\section{Struttura del progetto}

Il progetto è suddiviso nei seguenti package:

\begin{description}
	\item[server] contenente le classi necessarie all'implementazione del
	server di \emph{SimpleSocial}, ovvero:
	\begin{description}
		\item[Server] implementa l'eseguibile ed un'interfaccia basilare; offre
		alcuni metodi per il savataggio ed il caricamento della configurazione e
		del database degli utenti;
		\item[TCPClientHandler] implementa il thread che gestisce una
		connessione TCP proveniente da un client; esegue le principali
		funzionalità offerte dal server;
		\item[OnlineUsers] implementa il database degli utenti online; per ogni
		utente, memorizza (tramite una classe privata, {\bf LogInRecord}) il
		tempo di accesso, il token di autenticazione ({\tt oAuth}, un intero
		positivo generato pseudocasualmente al momento dell'accesso) e
		l'indirizzo IP del client;
		\item[UsersDB] implementa il database degli utenti iscritti; una sua
		istanza potrà essere serializzata e salvata su file, per essere
		ricaricata in seguito dal server;
		\item[User] rappresenta un utente di \emph{SimpleSocial}; per ogni
		utente memorizza username e password, la lista di amici, la lista di
		richieste di amicizia, la lista di follower e la lista di contenuti
		ancora non ricevuti;
		\item[PostDispatcher] notifica agli utenti quando viene pubblicato un
		nuovo contenuto da un utente da loro seguito (tramite RMI).
	\end{description}
	\item[client] contenente le classi necessarie all'implementazione del
	client di \emph{SimpleSocial}, ovvero:
	\begin{description}
		\item[Client] implementa l'eseguibile ed un'interfaccia utente di tipo
		testuale e fornisce alcuni metodi per interagire con il server, oltre a
		quelli per il salvataggio ed il caricamento della configurazione del
		client, del database delle richieste di amicizia non ancora confermate
		e del database dei {\bf Post} non ancora letti;
		\item[KeepAliveSignalHandler] implementa il thread dedicato alla
		ricezione e alla risposta dei messaggi di \emph{KeepAlive} mandati dal
		server;
		\item[NotificationHandler] implementa il thread dedicato alla ricezione
		di richieste di amicizia inoltrate dal server; delega ogni richiesta
		ad un'ulteriore thread, che esegue un {\bf Runnable} implementato dalla
		classe privata {\bf Handler};
		\item[PostNotifier] implementa una callback che verrà utilizzata da
		un'istanza di {\bf PostDispatcher} per mandare al client il contenuto
		di un {\bf Post} pubblicato da un amico.
	\end{description}
	\item[simplesocial] contenente i precedenti package {\bf server} e
	{\bf client}, più alcune classi da loro condivise:
	\begin{description}
		\item[Follower] stub per RMI (implementata poi da {\bf PostNotifier});
		\item[Updater] stub per RMI (implementata poi da {\bf PostDispatcher});
		\item[ErrorCode] {\tt enum} utilizzata per la rappresentazione di
		messaggi di errore (o di successo); offre metodi per la conversione da
		{\bf ErrorCode} a intero e viceversa;
		\item[Operation] {\tt enum} utilizzata per la comunicazione da parte del
		client al server per specificare il tipo di operazione da compiere;
		offre metodi per la conversione da {\bf Operation} a intero e viceversa;
		\item[Post] un contenuto pubblicato da un utente di \emph{SimpleSocial};
		ogni {\bf Post} memorizza il suo contenuto e l'utente che l'ha
		pubblicato.
	\end{description}
\end{description}

La \figurename~\ref{package} mostra le varie dipendenze tra le classi del
progetto.

\begin{figure}[h]
\begin{tikzpicture}
\begin{umlpackage}{simplesocial}
	\begin{umlpackage}[x=0, y=0]{server}
		\umlsimpleclass[x=-2, y=-4]{Server}
		\umlsimpleclass[x=0, y=-0]{TCPClientHandler}
		\umlsimpleclass[x=0.3, y=-1]{PostDispatcher}
		\umlsimpleclass[x=0.6, y=-2]{OnlineUsers}
		\umlsimpleclass[x=0.85, y=-3]{UsersDB}
		\umlsimpleclass[x=1.15, y=-4]{User}
		\umlaggreg[geometry=|-, anchor1=30]{Server}{UsersDB}
		\umluniassoc[geometry=|-, anchor1=150]{Server}{TCPClientHandler}
		\umluniassoc[geometry=|-, anchor1=120]{Server}{PostDispatcher}
		\umluniassoc[geometry=|-, anchor1=60]{Server}{OnlineUsers}
		\umluniassoc[geometry=-|-, arm1=0.5, anchor1=-10, anchor2=10]
				{UsersDB}{User}
		\umluniassoc[geometry=-|-, arm1=0.7, anchor1=-10, anchor2=-10]
				{OnlineUsers}{User}
		\umlaggreg[geometry=-|-, arm1=0.5, anchor1=-9, anchor2=10]
				{TCPClientHandler}{PostDispatcher}
		\umlaggreg[geometry=-|-, arm1=0.7, anchor1=0, anchor2=0]
				{TCPClientHandler}{OnlineUsers}
		\umlaggreg[geometry=-|-, arm1=1.3, anchor1=9, anchor2=0]
				{TCPClientHandler}{UsersDB}
		\umlaggreg[geometry=-|-, arm1=0.5, anchor1=-9, anchor2=10]
				{PostDispatcher}{OnlineUsers}
		\umlaggreg[geometry=-|-, arm1=1, anchor1=-1, anchor2=10]
				{PostDispatcher}{UsersDB}
	\end{umlpackage}
	\begin{umlpackage}[x=7, y=0]{client}
		\umlsimpleclass[x=-2, y=-2]{Client}
		\umlsimpleclass[x=1, y=0]{KeepAliveSignalHandler}
		\umlsimpleclass[x=1.4, y=-1]{NotificationHandler}
		\umlsimpleclass[x=2.1, y=-2]{PostNotifier}
		\umlassoc[geometry=|-, anchor1=145, anchor2=178, arg2=0..1, pos2=1.6]
		{Client}{KeepAliveSignalHandler}
		\umlassoc[geometry=|-, anchor1=65, anchor2=178, arg2=0..1, pos2=1.6]
		{Client}{NotificationHandler}
		\umlaggreg[geometry=-|, anchor1=185, anchor2=115]
		{KeepAliveSignalHandler}{Client}
		\umlaggreg[geometry=-|, anchor1=185, anchor2=35]
		{NotificationHandler}{Client}
		\umlassoc[arg2=0..1]{Client}{PostNotifier}
	\end{umlpackage}
	\umlsimpleclass[x=6.3, y=-5.5, type=enum]{ErrorCode}
	\umlsimpleclass[x=9.5, y=-5.5, type=enum]{Operation}
	\umlsimpleclass[x=9.5, y=-4, type=interface]{Follower}
	\umlsimpleclass[x=6.3, y=-4, type=interface]{Updater}
	\umlsimpleclass[x=3.7, y=-5.5]{Post}
	\umlimpl[geometry=-|]{PostNotifier}{Follower}
	\umlimpl[geometry=-|-, arm1=1.6, anchor1=5]{PostDispatcher}{Updater}
\end{umlpackage}
\end{tikzpicture}
\centering
\caption{Diagramma di package di \emph{SimpleSocial} \label{package}}
\end{figure}


\section{Eseguibili}

\subsection{Server}
Il server può essere eseguito dall'archivio {\tt SimpleServer.jar} con il
comando

\begin{verbatim}
	$ java -jar SimpleServer.jar [OPTION]...
\end{verbatim}

Questo comando accetta varie opzioni che permettono di modificare la
configurazione del server (ovvero impostare i valori delle varie porte, degli
indirizzi, ecc...). Le opzioni sono:

\begin{description}
	\item [{\tt -h, --help}] mostra le possibili opzioni e termina;
	\item [{\tt -d, --default}] utilizza i valori di default per tutti i
	parametri non specificati. Se non viene inserita questa opzione, il server
	caricherà la precedente configurazione salvata nel file {\tt conf.json}.
	La configurazione caricata potrà essere comunque modificata dalle altre
	opzioni, se specificate;
	\item [{\tt -s, --save}] salva la configurazione generata nel file
	{\tt conf.json} (al prossimo riavvio del server non sarà necessario passare
	nuovamente gli stessi argomenti);
	\item [{\tt -t, --tcp-port=PORT}] imposta {\tt PORT} come porta per le
	comunicazioni TCP;
	\item [{\tt -u, --udp-port=PORT}] imposta {\tt PORT} come porta per le
	comunicazioni UDP;
	\item [{\tt -m, --mc-port=PORT}] imposta {\tt PORT} come porta per le
	comunicazioni in multicast;
	\item [{\tt -r, --rmi-port=PORT}] imposta {\tt PORT} come porta usata dal
	registro RMI;
	\item [{\tt -k, --keep-alive=NUM}] imposta {\tt NUM} come intervallo di
	tempo tra un mesaggio di KeepAlive ed un altro (in millisecondi);
	\item [{\tt -g, --mc-group=IP}] imposta {\tt IP} come indirizzo del gruppo
	multicast;
	\item [{\tt -L, --log-duration=PERIOD}] imposta {\tt PERIOD}\footnote
	{{\tt PERIOD} deve essere rappresentato nel formato {\tt PnDTnHnMn.nS}
	(standard ISO-8601).} come periodo di validità di un accesso (dopo tale
	periodo l'accesso sarà considerato scaduto e l'utente dovrà rieffettuare
	il login);
	\item [{\tt -R, --req-duration=PERIOD}] imposta {\tt PERIOD}\footnotemark
	[\value{footnote}] come periodo di validità di una richiesta di amicizia
	(dopo tale periodo la richiesta non viene accettata non sarà più valida e
	dovrà essere rieffettuata).
\end{description}

Una volta avviato, il server non interagirà più con l'utente: i messaggi che
stamperà a video saranno solo quelli di avvenuta connessione da parte di un
client e di un'eventuale modifica del database degli utenti ({\bf UsersDB}).

\paragraph{File utilizzati} Il processo al suo avvio accede ai file
{\tt conf.json} (a meno che non venga passato come argomento {\tt -d}) e
{\tt users.db} presenti nella cartella {\tt server\_data/}.

Il primo contiene la configurazione del server in formato JSON, con i seguenti
attributi:

\begin{description}
	\item[{\tt tcp\_port}] che rappresenta il numero di porta utilizzato per le
	comunicazioni TCP;
	\item[{\tt udp\_port}] che rppresenta il numero di porta utilizzato per le
	comunicazioni UDP;
	\item[{\tt mc\_port}] che rappresenta il numero di porta utilizzato per le
	comunicazioni in multicast;
	\item[{\tt rmi\_port}] che rappresenta il numero di porta utilizzato dal
	registro RMI;
	\item[{\tt mc\_group}] che rappresenta l'indirizzo IP del gruppo multicast;
	\item[{\tt keep\_alive\_millis}] che rappresenta l'intervallo di tempo tra
	un messaggio di	KeepAlive ed un altro;
	\item[{\tt log\_duration}] che rappresenta il tempo di validità di un
	accesso;\footnote{Rappresentato nel formato {\tt PnDTnHnMn.nS} (standard
	ISO-8601).}
	\item[{\tt request\_duration}] che rappresenta il tempo di validità di una
	richiesta di amicizia.\footnotemark[\value{footnote}]
\end{description}

Il secondo contiene un oggetto {\bf UsersDB} serializzato, salvato su file al
termine di ogni esecuzione e ripristinato ad ogni avvio del server, per
garantire che i dati di ogni utente non vadano persi tra un'esecuzione e l'altra
(viene salvato anche in caso di terminazione improvvisa).\footnote{Questo è
anche il motivo principale per cui {\bf UsersDB} e {\bf OnlineUsers} sono due
entità separate: poteva essere infatti conveniente creare un'unica classe che
gestisse sia gli utenti registrati che quelli online (ed evitare così delle
ridondanze), ma una volta termitata l'esecuzione, i dati degli utenti online non
sarebbero stati più rilevanti, rendendo così inutile il salvataggio e ancor di
più il ripristino di tali informazioni.}

\paragraph{Thread e concorrenza} Ogni richiesta ricevuta dal server da parte di
un client viene affidata ad un nuovo thread {\bf TCPClientHandler}. Questo crea
un alto grado di concorrenza nelle strutture dati condivise, ovvero
{\bf UsersDB}, {\bf OnlineUsers} e {\bf PostDispatcher}. Questa concorrenza
viene gestita nei metodi delle classi stesse, mediante l'utilizzo di blocchi di
mutua esclusione alle strutture dati interne.

\subsection{Client}
Il client può essere eseguito dall'archivio {\tt SimpleClient.jar} con il
comando

\begin{verbatim}
	$ java -jar SimpleClient.jar
\end{verbatim}

Questo comando non accetta ulteriori argomenti, ma una volta eseguito chiede
all'utente quali azioni eseguire mediante la selezione di alcune opzioni.
Appena avviato il programma chiede se effettuare il login, registrarsi,
configurare il client o uscire. Nei primi due casi, dopo aver inserito username
e password, se il login o la registraizione\footnote{Per motivi pratici, la
registrazione esegue in automatico anche il login.} vanno a buon fine, mostra le
possibili interrogazioni che possono essere effettuate al server (ovvero
ricercare utenti, aggiungere amici, accettare richieste di amicizia, mostrare la
lista amici, pubblicare un contenuto o seguire un amico) o in locale (mostrare i
contenuti pubblicati dagli amici seguiti dall'utente o mostrare le richieste in
sospeso\footnote{Questa funzione viene effettuata automaticamente quando viene
richiesto di accettare una richiesta amicizia: prima mostra le richieste in
sospeso (in locale) poi chiede di inserire il nome dell'amico da accettare e
invia la richiesta al server.}). Se si decide di configurare il client, invece,
il programma offre un altro menù a scelta multipla con cui offre la possibilità
di modificare i valori dell'IP del server o del gruppo multicast, delle varie
porte utilizzate dal server (TCP, UDP, RMI, multicast), o selezionare
l'interfaccia di rete da utilizzare per la ricezione dei messaggi in
multicast\footnote{Questa opzione è stata inserita poichè
{\tt NetworkInterface.getByInetAddress( InetAddress.getLocalHost() )} può
restituire {\tt null} a seconda della macchina su cui viene eseguito il
programma. Il valore di default di questo attributo è {\tt wlan0}. Se questo
valore non dovesse andar bene, il thread {\bf KeepAliveSignalHandler} non
potrebbe ricevere i messaggi di KeepAlive dal server, causando così il logout
automatico dopo 10 secondi (default). Si consiglia quindi di modificare questo
valore al primo avvio, provando con {\tt eth0} o con una delle interfacce
mostrate col comando {\tt \$ ifconfig -a} (su sistema UNIX).}

\paragraph{File utilizzati} Il processo al suo avvio accede al file
{\tt conf.json} presente nella cartella {\tt client\_data/}, contenente la
configurazione del client in formato JSON, con i seguenti attributi:

\begin{description}
	\item[{\tt tcp\_port}] che rappresenta il numero di porta utilizzato dal
	server per le comunicazioni TCP;
	\item[{\tt udp\_port}] che rppresenta il numero di porta utilizzato dal
	server per le comunicazioni UDP;
	\item[{\tt mc\_port}] che rappresenta il numero di porta utilizzato dal
	server per le comunicazioni in multicast;
	\item[{\tt rmi\_port}] che rappresenta il numero di porta utilizzato dal
	registro RMI del server;
	\item[{\tt mc\_group}] che rappresenta l'indirizzo IP del gruppo multicast;
	\item[{\tt server\_address}] che rappresenta l'indirizzo IP del server;
	\item[{\tt network\_interface}] che rappresenta l'interfaccia di rete
	utilizzata dal client.
\end{description}

Inoltre, ad ogni login, il programma accede ai file
{\tt requests-<\emph{username}>.db} e {\tt posts-<\emph{username}>.db}, che
contengono rispettivamente il database delle richieste di amicizia non ancora
accettate (un oggetto {\bf CopyOnWriteArrayList<String>} serializzato) e il
database dei contenuti ricevuti non ancora letti (un oggetto
{\bf BlockingQueue<Post>} serializzato) appartenenti all'utente che ha
effettuato il login, il cui nome utente è {\tt \emph{username}}. Questi file
verranno poi sovrascritti al momento del logout. L'utilizzo del nome utente nel
nome dei file garantisce la sua unicità, poichè non possono esistere due utenti
con lo stesso nome. I file inoltre vengono modificati solamente se il login ha
successo, quindi un utente che non conosce la password non può accedere ai
contenuti dei file mediante l'utilizzo di questo programma.

\paragraph{Thread e concorrenza} Al momento del login, il processo crea due
nuovi thread: un {\bf KeepAliveSignalHandler} ed un {\bf NotificationHandler}.
Il primo ha come unico compito quello di rispondere ai messaggi di KeepAlive da
parte del server e non crea quindi grossi problemi di concorrenza. Il secondo,
invece, deve attendere da parte dal server l'arrivo di una richiesta di amicizia
da parte di un altro utente. Una volta arrivata, il thread crea a sua volta un
altro mini-thread e torna in attesa di un'altra richiesta. Il mini-thread, nel
frattempo, accetta la connessione dal server e salva la richiesta di amicizia
nell'apposita struttura dati del client. Per risolvere il problema della
concorrenza, viene utilizzata come struttura dati una
{\bf CopyOnWriteArrayList}, che è thread-safe.

Al momento del login, inoltre, viene creato un oggetto {\bf PostNotifier} che
verrà utilizzato come callback dal server per inviare al client i contenuti
degli utenti seguiti. Questo oggetto modifica il database dei contenuti non
ancora letti dal client, creando così un'altra concorrenza. Anche in questo caso
viene risolta con l'utilizzo di una struttura dati thread-safe, ovvero
{\bf BlockingQueue} (viene utilizzata una coda per mantenere l'ordine di arrivo
dei contenuti).

\subsection{Test Suite}
È possibile inoltre eseguire una suite di test sulle classi usate dal client e
dal server tramite l'utilizzo dell'archivio eseguibile {\tt TestSuite.jar} con
il comando

\begin{verbatim}
	$ java -jar TestSuite.jar
\end{verbatim}

Questo comando esegue tutte le classi di test presenti nella cartella
{\tt test/}, ovvero

\begin{description}
	\item[ErrorCodeTest] esegue un test di conversione dell'{\tt enum}
	{\bf ErrorCode};
	\item[OperationTest] esegue un test di conversione dell'{\tt enum}
	{\bf Operation};
	\item[UserTest] esegue il test sui vari metodi offerti dalla classe
	{\bf User};
	\item[UsersDBTest] esegue il test sui vari metodi offerti dalla classe
	{\bf UsersDB};
	\item[PostDispatcherTest] esegue il test sui vari metodi offerti dalla classe
	{\bf PostDispatcher};
	\item[ServerTest] esegue il test di salvataggio della configurazione e della
	serializzazione del database degli utenti;
	\item[ClientTest] esegue un test generale di interazione\footnote{In alcuni
	punti del test vengono utilizzate delle {\tt Thread.sleep()}. Queste non
	servono a risolvere una concorrenza non gestita, bensì ad attendere che il
	client (o il server) abbia eseguito l'operazione richiesta (ad esempio:
	quando il client esegue una {\tt post()}, il server risponde con
	{\tt SUCCESS} quando ha ricevuto correttamente l'istruzione ma non quando
	ha terminato di eseguirla. Questo può causare un'errore se si eseguono
	un'operazione di {\tt post()} e una operazione di lettura dei contenuti in
	rapida successione, ma non se viene utilizzato con tempi di reazione
	\emph{umani}).} tra più istanze di {\bf Client} ed una di {\bf Server};
	vengono eseguite tutte le principali funzioni.
\end{description}

L'esecuzione di questo processo produrrà una serie di messaggi di errore
(causati di proposito). Al termine dell'esecuzione, il programma stamperà a
schermo {\tt All test were successful!} se tutti i test sono stati superati,
oppure {\tt FAIL:} seguito da un messaggio di errore se alcuni test non sono
stati superati.

\section{Comunicazione tra processi}

\subsection{Canali di comunicazione}
Le comunicazioni client-server avvengono tramite {\bf SocketChannel} per
l'esecuzione delle funzionalità principali (login, registrazione, ricerca degli
utenti, ecc...), {\bf DatagramChannel} per l'invio dei messaggi di KeepAlive in
multicast e per la ricezione delle eventuali risposte (client-server UDP), e
{\bf Socket} per l'inoltro delle richieste di amicizia dal server al
destinatario.

Nel server, i vari {\bf Channel} sono in modalità non-bloccante e vengono
smistati con l'utilizzo di un {\bf Selector}. Quando arriva una richiesta da
parte di un client, il server accetta la connessione la delega ad un nuovo
thread {\bf TCPClientHandler}. Il canale risultante viene lasciato in modalità
bloccante, per permette di implementare più facilmente un protocollo di
comunicazione tra il server ed il client.

\subsection{Protocollo di comunicazione}
I client ed il server si scambiano messaggi tramite l'utilizzo di
{\bf ByteBuffer}. I messaggi sono composti per lo più da interi e stringhe,
convertiti in array di byte. Ogni stringa è sempre preceduta dalla lunghezza che
occupa nel buffer (quindi, da un intero). Il contenuto dei messaggi varia a
seconda della funzionalità da eseguire:

\begin{description}
	\item[{\tt login()}, {\tt register()}] invia la codifica dell'operazione da
	eseguire in intero, seguita dall'username, la password, ed il numero di
	porta\footnote{La porta viene generata dal thread {\bf NotificationHandler}
	in modo dinamico. Questo permette di poter utilizzare più client su una
	stessa macchina.} in cui il client attende connessioni per l'inoltro di una
	richiesta di amicizia. Il server risponde con un intero che rappresenta un
	{\bf ErrorCode} se negativo, altrimenti l'{\tt oAuth} se positivo (il client
	ha effettuato il login con successo);
	\item[{\tt addFriend()}] invia la codifica dell'operazione da eseguire in
	intero, seguita dall'username, l'{\tt oAuth}, ed il nome dell'utente a cui
	inviare la richiesta di amicizia. A questo punto, se l'autenticazione è
	valida e l'utente è online, il server apre una socket al suo indirizzo
	tramite la porta da lui specificata e gli invia una stringa contenente
	l'username dell'utente che ha effettuato la richiesta. Dopodichè, in ogni
	caso, il server risponde con un intero che rappresenta la codifica di un
	{\bf ErrorCode}. La \figurename~\ref{addfriend} mostra il diagramma di
	sequenza relativo allo svolgimento di questa funzionalità;

	\begin{figure}[h]
	\begin{tikzpicture}
	\begin{umlseqdiag}
	\umlactor[x=0, class=Client]{client1}
	\umldatabase[x=3, class=Server]{server}
	\umlactor[x=9,class=NotificationHandler]{client2}
	\begin{umlcall}[op={open()}, return={accept()}]{client1}{server}
		\umlcreatecall[x=6.5, class=TCPClientHandler]{server}{ch}
	\end{umlcall}
	\begin{umlcall}[op={write()}, return={close()}, dt=5]{client1}{ch}
		\begin{umlfragment}
		\begin{umlcall}[op={connect()}, return={accept()}, dt=5]{ch}{client2}
			\umlcreatecall[x=11.5, class=Handler]{client2}{sh}
		\end{umlcall}
		\begin{umlcall}[op={write()}, return={close()}, dt=5]{ch}{sh}
			\begin{umlcallself}[op={add()}, dt=5]{sh}
			\end{umlcallself}
		\end{umlcall}
		\end{umlfragment}
		\begin{umlcall}[type=return, op={write()}]{ch}{client1}
		\end{umlcall}
	\end{umlcall}
	\end{umlseqdiag}
	\end{tikzpicture}
	\centering
	\caption{Diagramma di sequenza di {\tt addFriend()} \label{addfriend}}
	\end{figure}

	\item[{\tt confirmFriendship()}] invia la codifica dell'operazione da
	eseguire in intero, seguita dall'username, l'{\tt oAuth}, ed il nome
	dell'utente a cui confermare la richiesta di amicizia. Il server, dopo aver
	confermato la  richiesta, risponde con un intero che rappresenta la codifica
	di un {\bf ErrorCode}.
	\item[{\tt getFriendsList()}] invia la codifica dell'operazione da eseguire
	in intero, seguita dall'username e dall'{\tt oAuth}. Il server, dopo aver
	verificato l'autenticazione, manda un intero che rappresenta un
	{\tt ErrorCode} se negativo, altrimenti il numero di utenti trovati nella
	lista di amicizie. Dopodiché, seguiranno tutti i nomi degli amici trovati in
	coppia con il loro rispettivo stato, codificato ad intero: 0 se online, -1
	se offline. Se il numero degli utenti trovato è alto, le coppie utente-stato
	possono essere distribuite più su buffer. Il client saprà che ci saranno
	altri buffer da ricevere finche non avrà ricevuto un numero di utenti pari
	a quello specificato all'inizio della serie.
	\item[{\tt search()}] come {\tt getFriendList()}, ma senza la codifica dello
	stato attuale;
	\item[{\tt post()}] invia la codifica dell'operazione da eseguire in intero,
	seguita dall'username e dall'{\tt oAuth}. Il server, dopo aver verificato
	l'autenticazione, manda un intero che rappresenta un {\tt ErrorCode}. Se
	quest'ultimo è uguale a {\tt SUCCESS}, il client invia al server la stringa
	del contenuto da pubblicare (sempre preceduta dalla sua lunghezza). Il
	contenuto può essere diviso su più buffer all'occorrenza;
	\item[{\tt logout()}] invia la codifica dell'operazione da eseguire in
	intero, seguita dall'username e dall'{\tt oAuth}. Il server, dopo aver
	eseguito il logout, risponde con un intero che rappresenta la codifica di un
	{\bf ErrorCode}.
\end{description}


% \begin{figure}[h]
% \begin{tikzpicture}
% \begin{umlseqdiag}
% \umlactor[x=0, class=Client]{client1}
% \umldatabase[x=3, class=Server]{server}
% \begin{umlcall}[op={open()}, return={accept()}]{client1}{server}
% 	\umlcreatecall[x=6.5, class=TCPClientHandler]{server}{ch}
% \end{umlcall}
% \begin{umlcall}[op={write()}, return={close()}, dt=5]{client1}{ch}
% 	\begin{umlfragment}[type=alt, label=success, inner xsep=6]
% 		\begin{umlfragment}[type=loop]]
% 		\begin{umlcall}[type=return, op={write()}, dt=5]{ch}{client1}
% 		\end{umlcall}
% 		\end{umlfragment}
% 	\umlfpart[default]
% 	\begin{umlcall}[type=return, op={write()}, dt=-2]{ch}{client1}
% 	\end{umlcall}
% 	\end{umlfragment}
% \end{umlcall}
% \end{umlseqdiag}
% \end{tikzpicture}
% \centering
% \caption{Diagramma di sequenza di {\tt search()} e {\tt getFriendsList()}
% \label{search}}
% \end{figure}



\section{Note finali}

Il progetto è stato sviluppato con IntelliJ IDEA 2016.2 (è possibile trovare i
file relativi al progetto all'interno della cartella {\tt SimpleSocial/}). Sono
state utilizzate le librerie standard Java (JDK 8 -- Oracle), JSON-Simple 1.1.1,
e JUnit4 per lo sviluppo della suite di test.

Ulteriori informazioni riguardo al progetto possono essere trovate nei vari
commenti all'interno dei file sorgente presenti in {\tt src/simplesocial/},
oppure consultando il JavaDoc presente in {\tt doc/html/index.html}.

\paragraph{Possibili errori} Eseguendo il test può capitare (in modo
apparentemente casuale) che nei punti dei metodi di {\bf Client} dove vengono
effettuate più letture dal server o più scritture al server in successione (ad
esempio in {\tt getFriendList()}, {\tt search()} o {\tt post()}) venga lanciata
un'eccezione del tipo {\bf BufferUnderflowException}\footnote{Questa eccezione
viene lanciata quando, utilizzando un {\bf ByteBuffer}, l'indicatore
{\tt position} va oltre l'indicatore {\tt limit}.}. Pare quindi che qualche
volta il server (o il client) non invii tutti i byte necessari a completare il
messaggio, causando così il lancio di questa eccezione.

\end{document}
